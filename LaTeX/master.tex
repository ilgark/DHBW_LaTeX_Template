%%%%%%%%%%%%%%%%%%%%%%%%%%%%%%%%%%%%%%%%%%%%%%%%%%%%%%%%%%
%   Autoren:
%   Prof. Dr. Bernhard Drabant
%   Prof. Dr. Dennis Pfisterer
%   Prof. Dr. Julian Reichwald
%%%%%%%%%%%%%%%%%%%%%%%%%%%%%%%%%%%%%%%%%%%%%%%%%%%%%%%%%%

%%%%%%%%%%%%%%%%%%%%%%%%%%%%%%%%%%%%%%%%%%%%%%%%%%%%%%%%%%
%	ANLEITUNG: 
%   1. Ersetzen Sie firmenlogo.jpg im Verzeichnis img
%   2. Passen Sie alle Stellen im Dokument an, die mit 
%      @stud 
%      markiert sind
%%%%%%%%%%%%%%%%%%%%%%%%%%%%%%%%%%%%%%%%%%%%%%%%%%%%%%%%%%

%%%%%%%%%%%%%%%%%%%%%%%%%%%%%%%%%%%%%%%%%%%%%%%%%%%%%%%%%%
%	ACHTUNG: 
%   Für das Erstellen des Literaturverzeichnisses wird das 
%   modernere Paket biblatex in Kombination mit biber 
%   verwendet - nicht mehr das ältere Paket BibTex!
%
%   Bitte stellen Sie Ihre TeX-Umgebung entsprechend ein (z.B. TeXStudio): 
%   Einstellungen --> Erzeugen --> Standard Bibliographieprogramm: biber
%%%%%%%%%%%%%%%%%%%%%%%%%%%%%%%%%%%%%%%%%%%%%%%%%%%%%%%%%%

\documentclass[fontsize=12pt,BCOR=5mm,DIV=12,parskip=half,listof=totoc,
               paper=a4,toc=bibliography,pointlessnumbers]{scrreprt}
               
               %toc=listof,listof=entryprefix,
               
\makeindex

%% Elementare Pakete, Konfigurationen und Definitionen werden geladen (gegebenenfalls anpassen)
\input{config}

%%
%% @stud
%%
%% PERSÖNLICHE ANGABEN (BITTE VOLLSTÄNDIG EINGEBEN zwischen den Klammern: {...})
%%
\ArtDerArbeit{<Bachelor>/<Projekt>} % "Bachelor" oder "Projekt" wählen
\TitelDerArbeit{<Titel Ihrer Arbeit>}
\AutorDerArbeit{<Ihr Vor- und Nachname>}
\Abteilung{<Ihre Abteilung>}
\Firma{<Ihre Firma>}
\Kurs{<Ihr Kurs>}
\Studienrichtung{<Ihre Studienrichtung>}
\Matrikelnummer{<Ihre Martikelnummer>}
\Studiengangsleiter{<Ihr Studiengangsleiter>}
\WissBetreuer{<Ihr(e) wissenschaftliche(r) Betreuer(in)>}
\FirmenBetreuer{<Ihr(e) Firmenbetreuer(in)>}
\Bearbeitungszeitraum{dd.mm.yyyy -- DD.MM.YYYY}
\Abgabedatum{dd.mm.yyyy}

%%
%% @stud
%%
%% BIBLIOGRAPHY (@stud: Bibliographie-Stil wählen - Position und Indizierung)
%%  Auswahl zwischen: 
%%   NUMERIC Style
%%   IEEE Style
%%   ALPHABETIC Style
%%   HARVARD Style
%%   CHICAGO Style 
%%   (oder eigenen zulässigen Stil wählen) 
%%
%%%%%%%%%%%%%
%% Zitierstil
%%%%%%%%%%%%%
% NUMERIC Style - e. g. [12]
%\newcommand{\indextype}{numeric} 
%
% IEEE Style - numeric kind of style 
%\newcommand{\indextype}{ieee} 
%
% ALPHABETIC Style - e. g. [AB12]
\newcommand{\indextype}{alphabetic} 
%
% HARVARD Style 
%\newcommand{\indextype}{apa} 
%
% CHICAGO Style 
%\newcommand{\indextype}{authoryear}
%
%%%%%%%%%%%%%%%%%%%%%%
%% Position des Zitats
%%%%%%%%%%%%%%%%%%%%%%
\newcommand{\position}{inline} 
%
% (!!) FOOTNOTE POSITION NOT RECOMMENDED IN MINT DOMAIN:
%\newcommand{\position}{footnote}

%% Final: Setzen des Zitierstils und der Zitatposition
\usepackage[backend=biber, autocite=\position, style=\indextype]{biblatex} 	
\settingBibFootnoteCite

%%
%% Definitionen und Commands
%%
\newcommand{\abs}{\par\vskip 0.2cm\goodbreak\noindent}
\newcommand{\nl}{\par\noindent}
\newcommand{\mcl}[1]{\mathcal{#1}}
\newcommand{\nowrite}[1]{}
\newcommand{\NN}{{\mathbb N}}
\newcommand{\imagedir}{img}

\makeindex

\begin{document}

\setTitlepage

%%%%%%%%%%%%%%%%%%%%%%%%%%%%%%%%%%%
% EHRENWÖRTLICHE ERKLÄRUNG
%
% @stud: ewerkl.tex bearbeiten
%
%\input{ewerkl} 
%\cleardoublepage  
%%%%%%%%%%%%%%%%%%%%%%%%%%%%%%%%%%%

%%%%%%%%%%%%%%%%%%%%%%%%%%%%%%%%%%%
% SPERRVERMERK
%
% @stud: nondisclosurenotice.tex bearbeiten
%
%\input{nondisclosurenotice} 
%\cleardoublepage
%%%%%%%%%%%%%%%%%%%%%%%%%%%%%%%%%%%

%%%%%%%%%%%%%%%%%%%%%%%%%%%%%%%%%%%
%	KURZFASSUNG
%
% @stud: acknowledge.tex bearbeiten
%
%\input{acknowledge}
%\cleardoublepage 
%%%%%%%%%%%%%%%%%%%%%%%%%%%%%%%%%%%

%%%%%%%%%%%%%%%%%%%%%%%%%%%%%%%%%%%
% VERZEICHNISSE und ABSTRACT
%
% @stud: ggf. nicht benötigte Verzeichnisse auskommentieren/löschen
%
\tableofcontents
\cleardoublepage

% Abbildungsverzeichnis
\phantomsection
\addcontentsline{toc}{chapter}{\listfigurename}
\listoffigures
\cleardoublepage

%	Tabellenverzeichnis
\phantomsection
\addcontentsline{toc}{chapter}{\listtablename}
\listoftables
\cleardoublepage

%	Listingsverzeichnis / Quelltextverzeichnis
% \lstlistoflistings
% \cleardoublepage

% Algorithmenverzeichnis
% \listofalgorithms
% \cleardoublepage

% Abkürzungsverzeichnis
% @stud: acronyms.tex bearbeiten
\input{acronyms}
\cleardoublepage

%	Kurzfassung / Abstract
% @stud: abstract.tex bearbeiten
% \input{abstract}
% \cleardoublepage

%%%%%%%%%%%%%%%%%%%%%%%%%%%%%%%%%%%%%%%%%%%%%%%%%%%%%%%%%%%%%%%%%%%%%%%%%%%%%%%%%%%%%%%%%%
% KAPITEL UND ANHÄNGE
%
% @stud:
%   - nicht benötigte: auskommentieren/löschen
%   - neue: bei Bedarf hinzufügen mittels input-Kommando an entsprechender Stelle einfügen
%%%%%%%%%%%%%%%%%%%%%%%%%%%%%%%%%%%%%%%%%%%%%%%%%%%%%%%%%%%%%%%%%%%%%%%%%%%%%%%%%%%%%%%%%%

\initializeText
\onehalfspacing

%%%%%%%%%%%%%%%%%%%%%%%%%%%%%%%%%%%
% KAPITEL
%
% @stud: einzelne Kapitel bearbeiten und eigene Kapitel hier einfügen
%
% Einleitung
% !TEX root =  master.tex
\chapter{Einleitung}

\nocite{*}


\section{Beispiel Abschnitt: \LaTeX-Installation}



\subsection{Beispiel Unterabschnitt: Aufbau eines \LaTeX-Dokuments}



% mehrere Grundlagen- und Forschungs-Kapitel
\input{chapter1}
% !TEX root =  master.tex
\chapter{Beispiel-Kapitel: Noch ein Kapitel}

blabla

\section{Abschnitt mit Coding}



% Fazit und Ausblick
% !TEX root =  master.tex
\chapter{Zusammenfassung}

\nocite{*}

Dieses Kapitel enthält die Zusammenfassung der Arbeit mit Fazit und Ausblick.


%%%%%%%%%%%%%%%%%%%%%%%%%%%%%%%%%%%

%%%%%%%%%%%%%%%%%%%%%%%%%%%%%%%%%%%
% ANHÄNGE
%
% @stud: einzelne Anhänge bearbeiten und eigene Anhänge hier einfügen 
%        die nachfolgenden Zeilen deaktivieren, wenn keine Anhänge verwendet werden
%
% \initializeAppendix
% \input{appendix1}
%%%%%%%%%%%%%%%%%%%%%%%%%%%%%%%%%%%

\singlespacing

%%%%%%%%%%%%%%%%%%%%%%%%%%%%%%%%%%%
% LITERATURVERZEICHNIS
% @stud: Literaturverzeichnis in Datei bibliography.bib anpassen. 
%
% Alternative zu Verwendung von \initializeBibliography: Citavi ...
% (dann \initializeBibliography auskommentieren und eigenes LaTex Coding verwenden)
%
\initializeBibliography
%%%%%%%%%%%%%%%%%%%%%%%%%%%%%%%%%%%

%%%%%%%%%%%%%%%%%%%%%%%%%%%%%%%%%%%
% INDEX
% @stud: ggf. Index auskommentieren, wenn nicht benötigt
%
% \addcontentsline{toc}{chapter}{Index}
% \printindex

\end{document}
